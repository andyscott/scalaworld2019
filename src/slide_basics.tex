\documentclass[include/preamble.tex]{subfiles}
\begin{document}

\tikzset{
    invisible/.style={opacity=0},
    visible on/.style={alt={#1{}{invisible}}},
    alt/.code args={<#1>#2#3}{%
      \alt<#1>{\pgfkeysalso{#2}}{\pgfkeysalso{#3}}%
  }
}

\newsavebox{\boxA}
\newsavebox{\boxB}
\newsavebox{\boxC}
\newsavebox{\boxD}

\begin{frame}[fragile]
  \begin{center}
    \begin{flushleft}
      imagine...
    \end{flushleft}
    \begin{itemize}
      \pause
    \item a list of strings
      \pause
    \item computing the total length of all of the strings
    \end{itemize}
  \end{center}
\end{frame}

\begin{frame}[fragile]
  \begin{center}
    \begin{tikzcd}[matrix scale=2.5, nodes={scale=1.25}]
      {List[String]}
      \arrow[visible on={<4->}, r, "f"]
      \ar[visible on=<6-7>, rr, bend right, shift right, swap, "h"]
      \ar[visible on=<8->, rr, bend right, shift right, swap, "g \circ f"]
      & |[visible on=<4->]| {List[Int]}
      \arrow[visible on=<5->, r, "g"]
      & |[visible on=<2->]| Int
    \end{tikzcd}
    \newline
    \newline
    \begin{flushleft}

      \begin{lrbox}{\boxA}
        \begin{lstlisting}[style=scala]
def f(x: List[String]): List[Int] = in.map(_.length)
        \end{lstlisting}
      \end{lrbox}

      \begin{lrbox}{\boxB}
        \begin{lstlisting}[style=scala]
def g(x: List[Int]): Int = in.foldLeft(0)(_ + _)
        \end{lstlisting}
      \end{lrbox}

      \begin{lrbox}{\boxC}
        \begin{lstlisting}[style=scala]
def h(x: List[String]): Int = g(f(x))
        \end{lstlisting}
      \end{lrbox}

      \begin{lrbox}{\boxD}
        \begin{lstlisting}[style=scala]
                     /* or */ g compose f
        \end{lstlisting}
      \end{lrbox}

      \visible<3->{\usebox{\boxA}}
      \newline
      \visible<5->{\usebox{\boxB}}
      \newline
      \newline
      \visible<7->{\usebox{\boxC}}
      \newline
      \visible<8->{\usebox{\boxD}}
    \end{flushleft}
  \end{center}
\end{frame}

\end{document}
