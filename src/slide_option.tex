\documentclass[include/preamble.tex]{subfiles}
\begin{document}

\begin{frame}[fragile]
  \frametitle{Folding Option}
  \begin{center}
    \begin{tikzpicture}[commutative diagrams/every diagram]
      \matrix[
        matrix of math nodes,
        matrix xscale=3,
        matrix yscale=3,
        nodes={scale=1.5},
        name=m,
        commutative diagrams/every cell
      ] {
        \only<10>{Unit}
        \only<11->{1}
        &
        \only<-12>{Option[A]}
        \only<13->{1 + A}
        &
        \only<8->{A}
        \\
        \only<6->{A}
        &
        \temporal<3-9>{}{Either[A, B]}{A + B}
        &
        \only<6->{B}
        \\
      };
      \path[commutative diagrams/.cd, every arrow, every label]
      (m-2-1) edge["\it{left}", visible on=<6->] (m-2-2)
      (m-2-3) edge["\it{right}", swap, visible on=<6->] (m-2-2)

      (m-1-1) edge["\it{none}", visible on=<10->] (m-1-2)
      (m-1-3) edge["\it{some}", swap, visible on=<8->] (m-1-2)
      ;

      \node[anchor=north west, visible on=<2-6>] at (m-1-1.south west) {
        \begin{minipage}{\textwidth}
          \begin{lstlisting}[style=scala]
sealed trait Option[A]
case class  Some[A](value: A) extends Option[A]
case object None extends Option[Nothing]
          \end{lstlisting}
        \end{minipage}
      };
      \node[anchor=north west, visible on=<7-8>] at (m-1-1.south west) {
        \begin{minipage}{\textwidth}
          \begin{lstlisting}[style=scala]
def none[A]      : Option[A] = None
def some[A](a: A): Option[A] = Some(a)
          \end{lstlisting}
        \end{minipage}
      };
      \node[anchor=north west, visible on=<9-11>] at (m-1-1.south west) {
        \begin{minipage}{\textwidth}
          \begin{lstlisting}[style=scala]
def none[A](u: Unit): Option[A] = None
def some[A](a: A): Option[A] = Some(a)
          \end{lstlisting}
        \end{minipage}
      };
      \node[anchor=north west, visible on=<12->] at (m-1-1.south west) {
        \begin{minipage}{\textwidth}
          \begin{lstlisting}[style=scala]
def none[A]      : Option[A] = None
def some[A](a: A): Option[A] = Some(a)
          \end{lstlisting}
        \end{minipage}
      };

      \node[anchor=north west, visible on=<4>] at (m-2-1.south west) {
        \begin{minipage}{\textwidth}
          \begin{lstlisting}[style=scala]
sealed trait Either[A, B]
case class Left [A, B](a: A) extends Either[A, B]
case class Right[A, B](b: B) extends Either[A, B]
          \end{lstlisting}
        \end{minipage}
      };

      \node[anchor=north west, visible on=<5->] at (m-2-1.south west) {
        \begin{minipage}{\textwidth}
          \begin{lstlisting}[style=scala]
def left [A, B](a: A): Either[A, B] = Left(a)
def right[A, B](b: B): Either[A, B] = Right(b)
          \end{lstlisting}
        \end{minipage}
      };

    \end{tikzpicture}
  \end{center}
\end{frame}


\begin{frame}[fragile]
  \frametitle{Folding Option}
  \begin{center}
    \begin{tikzpicture}[commutative diagrams/every diagram]
      \matrix[
        matrix of math nodes,
        matrix xscale=3,
        matrix yscale=1,
        nodes={scale=1.5},
        name=m,
        commutative diagrams/every cell
      ] {
        A
        &
        {Option[A]}
        \\
        1
        &
        {Option[A]}
        \\
        \only<2->{1 + A}
        &
        \only<2->{Option[A]}
        \\
        \only<3->{1 + A}
        &
        \only<3->{Option[A]}
        \\
        \only<4->{1 + A}
        &
        \only<4->{Option[A]}
        \\
      };
      \path[commutative diagrams/.cd, every arrow, every label]
      (m-1-1) edge["\it{some}"] (m-1-2)
      (m-2-1) edge["\it{none}"] (m-2-2)
      (m-3-1) edge["$\lbrack{empty, some}\rbrack$", visible on=<2->] (m-3-2)
      (m-4-2) edge["\textit{match}", swap, visible on=<3->] (m-4-1)
      ;
    \end{tikzpicture}
  \end{center}
\end{frame}

\begin{frame}[fragile]
  \begin{center}
    \begin{lstlisting}[style=scala]
def fold[A, B](x: Option[A])
              (ifEmpty: B)(f: A => B): B = ???
    \end{lstlisting}
    \pause
    \vspace{1em}
    \begin{lstlisting}[style=scala]
val x0  : Option[Int] = Some(2)
val res0: Int         = x0.fold(0)(_ * 10)
// 20

val x1  : Option[Int] = None
val res1: Int         = x1.fold(0)(_ * 10)
// 0
    \end{lstlisting}
  \end{center}
\end{frame}

\begin{frame}[fragile]
  \begin{center}
    \begin{lstlisting}[style=scala]
val x0  : Option[Int] = Some(2)
val res0: Int         = x0.fold(0)(_ * 10)
// 20

val x1  : Option[Int] = None
val res1: Int         = x1.fold(0)(_ * 10)
// 0

def fold[A, B](x: Option[A])
              (ifEmpty: B)(f: A => B): B =
  x match {
    case None    => ifEmpty
    case Some(a) => f(a)
  }
    \end{lstlisting}
  \end{center}
\end{frame}

\begin{frame}[fragile]
  \begin{center}
    \begin{lstlisting}[style=scala]
def fold[A, B](x: Option[A])
              (ifEmpty: B)(f: A => B): B =
  x match {
    case None    => ifEmpty
    case Some(a) => f(a)
  }
    \end{lstlisting}
    \begin{tikzcd}[matrix scale=2, nodes={scale=1.5}]
      {Option[A]} \arrow[rrr, "fold"] & & & B
      \\
      1 \arrow[rrr, "ifEmpty"] & & & B
      \\
      A \arrow[rrr, "f"] & & & B
    \end{tikzcd}
  \end{center}
\end{frame}

\begin{frame}[fragile]
  \begin{center}
    \begin{lstlisting}[style=scala]
def fold[A, B](x: Option[A])
              (ifEmpty: B)(f: A => B): B =
  x match {
    case None    => ifEmpty
    case Some(a) => f(a)
  }
    \end{lstlisting}
    \begin{tikzcd}[matrix scale=2, nodes={scale=1.5}]
      {Option[A]} \arrow[rrr, "fold"] & & & B
      \\
      1 + A \arrow[rrr, "\lbrack {ifEmpty, f} \rbrack"] & & & B
    \end{tikzcd}
  \end{center}
\end{frame}

\begin{frame}[fragile]
  \begin{center}
    \begin{lstlisting}[style=scala]
def fold[A, B](x: Option[A])
              (ifEmpty: B)(f: A => B): B =
  x match {
    case None    => ifEmpty
    case Some(a) => f(a)
  }
    \end{lstlisting}
    \begin{tikzcd}[matrix scale=2, nodes={scale=1.5}]
      {Option[A]}
      \arrow[rr, "fold"]
      \arrow[rd, swap, "match", visible on=<2->]
      & & B
      \\
        & 1 + A \arrow[ru, swap, "\lbrack {ifEmpty, f} \rbrack"] &
    \end{tikzcd}
  \end{center}
\end{frame}


\end{document}
