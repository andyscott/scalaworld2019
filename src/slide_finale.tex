\documentclass[include/preamble.tex]{subfiles}
\begin{document}

\tikzstyle{showborder}=[draw=black]

\begin{frame}[fragile]
  \frametitle{from foldRight \only<2->{to hylo}}
  \begin{center}
    \begin{lstlisting}[style=scala]
// foldRight just for list
def foldRight[A, B](
  la: List[A]
)(z: B)(f: (A, B) => B): B = la match {
  case Nil => z
  case head :: tail => f(head, foldRight(tail)(z)(f))
}
    \end{lstlisting}
    \pause
    \hrulefill
    \begin{lstlisting}[style=scala]
// 'foldRight' (and lots more)
// for any recursive data structure
def hylo[F[_]: Functor, A, B](a: A)(
  alg  : F[B] =>   B,
  coalg:   A  => F[A]
): B =
  alg(coalg(a).map(hylo(_)(alg, coalg)))
    \end{lstlisting}
  \end{center}
\end{frame}

\begin{frame}[fragile]
  \frametitle{from foldRight to hylo}
  \begin{center}
    \begin{lstlisting}[style=scala]
val input = List("hello", "scala", "world")

val z: Int = 0
val f: (String, Int) => Int = _.length + _

foldRight(input)(z)(f)
// 15
    \end{lstlisting}
    \pause
    \hrulefill
    \begin{lstlisting}[style=scala]
def alg: Option[(String, Int)] => Int =
  _ match {
    case None => 1
    case Some((head, acc)) => head.length + acc
  }

hylo(input)(alg, unpackList)
// 15
    \end{lstlisting}
  \end{center}
\end{frame}

\begin{frame}[fragile]
  \frametitle{from foldRight to hylo}
  \begin{center}
    \begin{itemize}
    \item
      We're not going to prove anything
    \item
      We're just going to use our understanding of Scala and
      foldRight to gain intuition for how recursion schemes work.
    \end{itemize}
  \end{center}
\end{frame}

\begin{frame}[fragile]
  \frametitle{creating/unpacking lists}
  \begin{center}
    \begin{tikzpicture}[commutative diagrams/every diagram]
      \matrix[
        matrix of math nodes,
        matrix xscale=3,
        matrix yscale=0.75,
        nodes={scale=1.5},
        name=m,
        commutative diagrams/every cell,
        nodes in empty cells
      ] {
        1
        &
        {List[A]}
        \\
        {A \times List[A]}
        &
        {List[A]}
        \\
        \\
        \\
        {1 + A \times List[A]}
        &
        {List[A]}
        \\
        {1 + A \times List[A]}
        &
        {List[A]}
        \\
      };
      \path[commutative diagrams/.cd, every arrow, every label]
      (m-1-1) edge["$nil$"] (m-1-2)
      (m-2-1) edge["$cons$"] (m-2-2)
      (m-5-1) edge["$\lbrack{nil, cons}\rbrack$"] (m-5-2)
      (m-6-2) edge["$match$", swap, zz] (m-6-1)
      ;

      \node[anchor=north west] at (m-5-1.west |- m-2-1.south) {
        \begin{minipage}{\textwidth}
          \begin{lstlisting}[style=scala]
def nil[A]: List[A] = Nil
def cons[A](head: A, tail: List[A]): List[A] =
  head :: tail
          \end{lstlisting}
        \end{minipage}
      };

    \end{tikzpicture}
  \end{center}
\end{frame}

\begin{frame}[fragile]
  \frametitle{List foldRight}
  \begin{center}
    \begin{lstlisting}[style=scala]
def foldRight[A, B](la: List[A])
  (z: B)(f: (A, B) => B): B = la match {
    case Nil => z
    case head :: tail => f(head, foldRight(tail)(z)(f))
  }
    \end{lstlisting}
    \pause
    \begin{tikzpicture}[commutative diagrams/every diagram]
      \matrix[
        matrix of math nodes,
        matrix xscale=3,
        matrix yscale=1.25,
        nodes={scale=1.5},
        name=m,
        commutative diagrams/every cell
      ] {
        \only<2->{List[A]}
        &
        \only<2->{B}
        \\
        \only<3-4>{1}
        \only<5>{1 + A \times B}
        &
        \only<3-5>{B}
        \\
        \only<4>{A \times B}
        \only<7->{1 + A \times List[A]}
        &
        \only<4>{B}
        \only<6->{1 + A \times B}
        \\
      };
      \path[commutative diagrams/.cd, every arrow, every label]
      (m-1-1) edge["$foldRight$", visible on=<2->] (m-1-2)
      (m-2-1) edge["$z$", visible on=<3-4>] (m-2-2)
      (m-3-1) edge["$f$", visible on=<4>] (m-3-2)
      (m-2-1) edge["$\lbrack{z, f}\rbrack$", visible on=<5>] (m-2-2)
      (m-3-2) edge["$\lbrack{z, f}\rbrack$", swap, visible on=<6->] (m-1-2)
      (m-1-1) edge["$match$", swap, zz, visible on=<7->] (m-3-1)
      (m-3-1) edge["$id + ...$", swap, visible on=<8>] (m-3-2)
      (m-3-1) edge["$id + id \times ...$", swap, visible on=<9>] (m-3-2)
      (m-3-1) edge["$id + id \times foldRight$", swap, visible on=<10>] (m-3-2)
      ;
    \end{tikzpicture}
  \end{center}
\end{frame}

\begin{frame}[fragile]
  \frametitle{foldRight intuition}
  \begin{center}
    \begin{lstlisting}[style=scala]
def foldRight[A, B](
  la: List[A]
)(z: B)(f: (A, B) => B): B = // ...

val list0 = "hello" :: "scala" :: "world" :: Nil
val f: (String, Int) => Int = _.length + _
foldRight(list0)(1)(f)
// 15
    \end{lstlisting}
    \pause
    \begin{lstlisting}[style=scala]
val list1 = ::("hello", ::("scala", ::("world", Nil)))
foldRight(list1)(1)(f)
// 15
    \end{lstlisting}
  \end{center}
\end{frame}

\begin{frame}[fragile]
  \frametitle{foldRight intuition}
  \begin{center}
    \lstset{aboveskip=0pt,belowskip=0pt}
    \begin{lstlisting}[style=scala]
::("hello", ::("scala", ::("world", Nil))) // list1
    \end{lstlisting}
    \pause
    \vspace{1em}
    \begin{lstlisting}[style=scala]
 f("hello",  f("scala",  f("world", 0)))   // 15
    \end{lstlisting}
    \pause
    \begin{lstlisting}[style=scala]
 f("hello",  f("scala",  5))               // 15
    \end{lstlisting}
    \pause
    \begin{lstlisting}[style=scala]
 f("hello",  10)                           // 15
    \end{lstlisting}
    \pause
    \begin{lstlisting}[style=scala]
 15                                        // 15 :)
    \end{lstlisting}
    \pause
    \vspace{1em}
    \begin{lstlisting}[style=scala]
 f(list.head,                // "hello".length +
   f(list.tail.head,         // "scala".length +
     f(list.tail.tail.head,  // "world".length +
       0)))                  // 0
    \end{lstlisting}
  \end{center}
\end{frame}

\begin{frame}[fragile]
  \frametitle{reshaping foldRight}
  \newcommand*{\emphaa}[1]{
    \tikz[
      baseline,
      outer sep=0pt, inner sep = 2pt
    ]
    \node[
      rectangle,
      onslide=<2>{showborder},
      anchor=text,
      nodes={scale=0.8},
    ]{#1};
  }
  \newcommand*{\emphab}[1]{
    \tikz[
      baseline,
      outer sep=0pt, inner sep = 2pt
    ]
    \node[
      rectangle,
      onslide=<2>{showborder},
      anchor=text
    ]{#1};
  }
  \newcommand*{\emphf}[1]{
    \tikz[
      baseline,
      outer sep=0pt, inner sep = 2pt,
      nodes={scale=0.8},
    ]
    \node[
      rounded rectangle,
      onslide=<2>{showborder},
      anchor=text
    ]{#1};
  }
  \begin{center}
    \begin{tikzpicture}[commutative diagrams/every diagram]
      \matrix[
        matrix of math nodes,
        matrix xscale=3,
        matrix yscale=3,
        nodes={scale=1.25},
        name=m,
        commutative diagrams/every cell
      ] {
        \only<-4>{\emphaa{$List[A]$}}
        \only<5->{A}
        &
        B
        \\
        \only<-3>{\emphf{$1 + A \times \emphab{List[A]}$}}
        \only<4>{F[List[A]]}
        \only<5->{F[A]}
        &
        \only<-3>{\emphf{$1 + A \times \emphab{B}$}}
        \only<4->{F[B]}
        \\
      };
      \path[commutative diagrams/.cd, every arrow, every label]
      (m-1-1) edge["$foldRight$", visible on=<-5>] (m-1-2)
      (m-1-1) edge["$hylo$", visible on=<6->] (m-1-2)
      (m-1-1) edge["$match$", swap, zz, visible on=<-8>] (m-2-1)
      (m-1-1) edge["$coalgebra$", swap, visible on=<9->] (m-2-1)
      (m-2-2) edge["$\lbrack{z, f}\rbrack$", swap, visible on=<-7>] (m-1-2)
      (m-2-2) edge["$algebra$", swap, visible on=<8->] (m-1-2)
      (m-2-1) edge["$id + id \times foldRight$", swap, visible on=<-6>] (m-2-2)
      (m-2-1) edge["$map\ hylo$", swap, visible on=<7->] (m-2-2)
      ;
    \end{tikzpicture}
  \end{center}
\end{frame}

\begin{frame}[fragile]
  \frametitle{hylo}
  \begin{center}
    \begin{tikzpicture}[commutative diagrams/every diagram]
      \matrix[
        matrix of math nodes,
        matrix xscale=3,
        matrix yscale=3,
        nodes={scale=1.5},
        name=m,
        commutative diagrams/every cell
      ] {
        A
        &
        B
        \\
        {F[A]}
        &
        {F[B]}
        \\
      };
      \path[commutative diagrams/.cd, every arrow, every label]
      (m-1-1) edge["$hylo$"] (m-1-2)
      (m-1-1) edge["$coalgebra$", swap] (m-2-1)
      (m-2-2) edge["$algebra$", swap] (m-1-2)
      (m-2-1) edge["$map\ hylo$", swap] (m-2-2)
      ;
    \end{tikzpicture}
    \begin{tikzpicture}
      \node[visible on=<1>] {
        \begin{minipage}{\textwidth}
          \begin{lstlisting}[style=scala]
def hylo[F[_]: Functor, A, B](a: A)(
  alg  : F[B] =>   B,
  coalg:   A  => F[A]
): B =
  //...
          \end{lstlisting}
        \end{minipage}
      };
      \node[visible on=<2>] {
        \begin{minipage}{\textwidth}
          \begin{lstlisting}[style=scala]
def hylo[F[_]: Functor, A, B](a: A)(
  alg  : F[B] =>   B,
  coalg:   A  => F[A]
): B =
      coalg(a)
          \end{lstlisting}
        \end{minipage}
      };
      \node[visible on=<3>] {
        \begin{minipage}{\textwidth}
          \begin{lstlisting}[style=scala]
def hylo[F[_]: Functor, A, B](a: A)(
  alg  : F[B] =>   B,
  coalg:   A  => F[A]
): B =
      coalg(a).map(hylo(...          ))
          \end{lstlisting}
        \end{minipage}
      };
      \node[visible on=<4>] {
        \begin{minipage}{\textwidth}
          \begin{lstlisting}[style=scala]
def hylo[F[_]: Functor, A, B](a: A)(
  alg  : F[B] =>   B,
  coalg:   A  => F[A]
): B =
      coalg(a).map(hylo(_)(alg, coalg))
          \end{lstlisting}
        \end{minipage}
      };
      \node[visible on=<5>] {
        \begin{minipage}{\textwidth}
          \begin{lstlisting}[style=scala]
def hylo[F[_]: Functor, A, B](a: A)(
  alg  : F[B] =>   B,
  coalg:   A  => F[A]
): B =
  alg(coalg(a).map(hylo(_)(alg, coalg)))
          \end{lstlisting}
        \end{minipage}
      };
    \end{tikzpicture}
  \end{center}
\end{frame}

\begin{frame}[fragile]
  \frametitle{hylo intuition}
  \begin{center}
    \begin{lstlisting}[style=scala]
val input = List("hello", "scala", "world")
val f: (String, Int) => Int = _.length + _

foldRight(input)(0)(f) // 15
    \end{lstlisting}
  \end{center}
\end{frame}

\begin{frame}[fragile]
  \frametitle{hylo intuition}
  \begin{center}
    \begin{lstlisting}[style=scala]
val input = List("hello", "scala", "world")

val unpackList: List[String] => Option[(String, List[String])] =
  _ match {
    case Nil          => None
    case head :: tail => Some((head, tail))
  }

val alg: Option[(String, Int)] => Int =
  _ match {
    case None              => 0
    case Some((head, acc)) => head.length + acc
  }

hylo(input)(alg, unpackList) // 15
    \end{lstlisting}
  \end{center}
\end{frame}

\begin{frame}[fragile]
  \frametitle{hylo intuition}
  \begin{center}
    \begin{lstlisting}[style=scala]
f(list.head,                // "hello".length +
  f(list.tail.head,         // "scala".length +
    f(list.tail.tail.head,  // "world".length +
      z)))                  // 0
    \end{lstlisting}
    \begin{lstlisting}[style=scala]
alg(coalg(list).fmap(ist => // "hello".length +
  alg(coalg(ist).fmap(st => // "scala".length +
    alg(coalg(st).fmap(t => // "world".length +
      alg(coalg(t).map(_ => // 0
        ???))))))))
    \end{lstlisting}
  \end{center}
\end{frame}

\end{document}
